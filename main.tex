\documentclass[format=acmsmall, screen=true, review=false]{acmart}
\usepackage{graphics}
\usepackage[dvips]{graphicx}
\usepackage[left]{lineno}
\usepackage[utf8]{inputenc}

\begin{document}

\begin{center}
%======
%Title
%======

{\bf Meta-Heurística aplicada a la optimización de ubicación de recursos policiales en Paraguay}
\bigskip

%============================
%List of Authors and Address
%============================

{\small
Luis Alberto Alvarez Penayo\footnote{E-mail: lalvarez.pol@gmail.com}, Marco Antonio Alvarez Penayo\footnote{E-mail: marcoalvarez@outlook.com}\\
\smallskip
Facultad Politécnica\\
Universidad Nacional de Asunción\\
San Lorenzo, Paraguay
}
\end{center}


%============================
%Abstract and Keywords
%============================

\begin{abstract}
All papers relevant to the topics of the JCIS are welcome. The abstract should emphasize what is new and important and contain as much detail of the work as possible. Please follow the format and style described in this template.  The abstract should be at least around 80 words.
Each Figure should be placed in the main text as near  as possible where it is discussed or, alternatively, all Figures can be placed preceding the reference list.

\bigskip

{\footnotesize
{\bf Keywords}: Objectivity, quality, perseverance, discernment.}
\end{abstract}

\section{Introdución}
La seguridad ciudadana constituye uno de los principales problemas sociales de varios países de América Latina, cuyos ciudadanos están profundamente preocupados por el fuerte incremento en la tasa de criminalidad, se sienten cada vez más inseguros en sus personas y bienes, y expresan su insatisfacción con respecto a la respuesta estatal ante el fenómeno delictivo\cite{SEGURIDAD-2002}.\\
El fenómeno delictivo es un tema complejo, y muchas veces parece que cualquier acción que se realice para prevenirlo o controlarlo resulta insuficiente. Para esto es fundamental desarrollar diagnósticos de calidad basados en indicadores válidos y confiables.  En este trabajo, vale la pena analizar la posibilidad de utilizar la georreferenciación de variables delictivas como una herramienta de alto valor en la focalización de acciones de prevención y control\cite{ENSC-2014}.\\
La Policia Nacional cuenta con jefaturas zonales y departamentales. Las jefaturas departamentales se subdividen en comisarías. La ciudad de Asunción, por ejemplo, cuenta con veinticuatro (24); éstas tienen a su cargo cuatro o cinco cuadrantes, en los que operan un vehículo con dos o tres policías. Cada cuadrante tiene un jefe policial que articula acciones con las comisiones vecinales de su jurisdicción, y en algunos casos una o varias casetas y sus propias cámaras de video-vigilancia. Se calcula que cada vehículo recorre diariamente entre 100 y 150 kilómetros. A su vez, cada cuadrante se subdivide en sectores\cite{GEOREFERENCIA-2005}.

\begin{thebibliography}{43}

\bibitem{SEGURIDAD-2002}
RICO J.M., CHINCHILLA L., 2002. Seguridad ciudadana en América Latina: hacia una política integral.

\bibitem{GEOREFERENCIA-2005}
ANDREAS HEIN, 2005. La Georreferenciación como herramienta para el diagnóstico de problemas de seguridad ciudadana en el ámbito local.

\bibitem{ENSC-2014}
REPÚBICA DEL PARAGUAY - MINISTERIO DEL INTERIOR, 2014. Estrategia Nacional de Seguridad Ciudadana (ENSC)

\bibitem{TABU-2014}
NILSON F. MATOS M., ANDRÉ G. DOS SANTOS, 2014. Heurística baseada em busca tabu para o posicionamento de unidades policiais.


\end{thebibliography}

\end{document}
