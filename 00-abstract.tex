\begin{abstract}
The response time to a 911 emergency calls is a very common problem in almost all developing countries. This paper seeks to optimize the response time in the displacement of police resources once they have been assigned to the incident, through a better geographic positioning taking into account the history of incidents that occurred over a period of time. As a basis for the tests, the city of Asunción and a history of two years of incidents were taken into account. The results showed that by strategically positioning the police resources, the response time to a emergency call could be considerably reduced, as well as reducing operating costs to the Paraguayan State.\\
%El tiempo de respuesta a una llamada de auxilio en las llamadas al 911 es un problema muy común en casi todos los países en desarrollo. En este trabajo se busca optimizar el tiempo de respuesta en el desplazamiento de los recursos policiales una vez que hayan sido asignados al incidente, mediante un mejor posicionamiento geográfico tomando en cuenta el historial de incidentes ocurridos en un lapso de tiempo. Como base para las pruebas se tomaron en cuenta la ciudad de Asunción y un historial de dos años de incidentes. Los resultados demostraron que mediante el posicionamiento estratégico de los recursos policiales se podría reducir considerablemente el tiempo de respuesta a una llamada de auxilio así como tambien reducir costos operativos al Estado paraguayo.\\
\bigskip
{\footnotesize{\bf Keywords}: Tiempo de respuesta, 911, Incidentes, Asunción.}
\end{abstract}