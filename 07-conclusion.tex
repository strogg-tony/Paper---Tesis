\section{Conclusión}
En este trabajo realizamos una implementación de la Búsqueda Tabú como propuesta de solución ante la problemática de ubicación óptima de recursos frente al modelo aleatorio utilizado actualmente.

En la evaluación de rendimiento, mostramos que la Búsqueda Tabú obtuvo en general mejores valores en cuanto a niveles de violencia cubiertos para los tipos de recursos patrulleras y motocicletas. En los casos de oficiales a pie no siempre se obtenían mejores valores, ésto debido a que el procedimiento de intensificación se encarga de optimizar el posicionamiento de recursos no motorizados en zonas donde no pueden ingresar vehículos.
Ambos modelos muestran un buen nivel de cobertura total para un tiempo máximo $T_{max}$ y $2T_{max}$.

A diferencia de la metodología actual, el trabajo propone que los recursos policiales (patrulleras, motocicletas y oficiales a pie) pertenezcan al Sistema Nacional de Emergencias 911 y no se encuentren adscritos a las comisarías, lo que facilita la libre ubicación de los mismos para optimizar las zonas críticas de cobertura.

Como trabajos futuros proponemos la evaluación de los delitos por franjas horarias de manera a tener diferentes posiciones de recursos acorde vaya transcurriendo las horas del día, ya que en algunas zonas el nivel de violencia aumenta o disminuye al movimiento de personas. Otro enfoque sería cambiar el algoritmo de la Búsqueda Tabú por el algoritmo Particle Swarm Optimization (PSO) o bien uno más reciente como el Bat-Algorithm.